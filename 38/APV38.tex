\documentclass[11pt,a4paper]{article}

\usepackage[utf8]{inputenc} 
\usepackage[T1]{fontenc} 
\usepackage{lmodern}
\usepackage[margin=2cm]{geometry}
\usepackage[german]{babel}

\setlength{\parindent}{0pt}
\setlength{\parskip}{1ex plus 0.5ex minus 0.5ex}

\usepackage{amsmath} 
\usepackage{graphicx} 
\usepackage{booktabs}
\usepackage[colorlinks]{hyperref}
\usepackage{nicefrac}
\usepackage{gensymb}

\begin{document}

{
\centering 
\large 
Physiklabor für Anf\"anger*innen \\
Ferienpraktikum im Sommersemester 2018 \\[4mm]
\textbf{\LARGE 
Versuch 38: Wärmekapazität
} \\[3mm]
(durchgef\"uhrt am 07.09.2018 bei Daniel Bartle) \\
Andréz Gockel, Patrick M\"unnich\\
\today \\[10mm]
}

\section{Ziel des Versuchs}

Der Versuch ist in zwei Teile geteilt, welche dazu dienen, die Wärmekapazität von Wasser zu bestimmen. Im Teil A bestimmt man die Temperatur Differenz von Wasser während der Umwandlung von mechanischer Energie zu thermischer Energie durch Reibkräfte mit Hilfe des Schürholz-Apparates. Im Teil B verwendet man einen elektrischen Widerstand um elektrische Leistung über ein bestimmten Zeitraum in thermische Energie zu wandeln.

\section{Auswertung und Fehleranalyse}

\subsection{Teil A - Umwandlung von mechanischer Arbeit in Wärme }
$$
\begin{array}{lc}
	\multicolumn{1}{l}{\textrm{Zu Bestimmende Werte}} \\
	\hline
	\textrm{Masse Wasser} & m_W \\
	\textrm{Masse Kalorimeter} & m_{kal} \\
	\textrm{Umdrehung} & n \\
	\textrm{Temperatur} & \Delta T\\
	\textrm{Durchmesser Kalorimeter} & d\\
\end{array}
\qquad
\begin{array}{p{3.5cm}l c}
	\multicolumn{2}{l}{\textrm{Bekannte Werte (Fehler nicht-beitragend)}} \\
	\hline
	\textrm{Spezifische Wärme\-kapazität Kupfer} & c_{Cu} = 0.38\, \nicefrac{\textrm{kJ}}{\textrm{(kgK)}} \\
	\textrm{Wärmekapazität vom Nylonseil} & C_T = 5\, \nicefrac{\textrm{J}}{\textrm{K}} \\
	\textrm{Masse Gewicht} & m = 5\, \textrm{kg} \\
\end{array}
$$



\subsubsection{Aufgabenstellung}

Mit Hilfe dem Sch\"urholz Apparat ist die W\"armekapazit\"at von Wasser zu bestimmen. Dies wird getan, indem man ein Nylonfaden \"uber ein mit $50$ ml Wasser gef\"ullten Kaloritmeter windet, ein Thermometer dranschraubt und am Ende des Nylonfadens eine $5$ kg Masse dranh\"angt. Das Kalorimeter wird gedreht, sodass die Feder vom Sch\"urholz Apparat entspannt ist, also die Reibkraft $F_R$ die Gewichtskraft $F_G$ ausgleicht. Die Temperatur wird beim Drehen gemessen und notiert.

%Picture from book

\subsubsection{Auswertung}

Die Messung wurde zweimal durchgef\"uhrt. Bei der ersten Durchf\"uhrung wurde die Temperatur\"anderung \"uber 50 Drehungen jeweils im Abstand von 10 Drehungen gemessen. Die zweite Durchf\"uhrung wurde mit 100 Drehungen durchgef\"uhrt und die Temperatur alle 5 Drehungen notiert.

%Table with values

$$
 \begin{array}{ll}
 	 \textrm{In Wasser}: \\ \textrm{werte in mm} \\ \textrm{Unsicherheit: $\pm 0.1$mm}
 \end{array}
%\rowcolors{2}{gray!10}{white}
\noindent%
\begin{tabular}{|l | ccccc|}
\hline
Messung & 1 & 2 & 3 & 4 & 5\\
\hline
Ruhelage $x_0$ & 441 & 479 & 473 & 463 & 468\\
Nicht eingetaucht $x_1$ & 414 & 453 & 447 & 436 & 440\\
Eingetaucht $x_2$ & 417 & 456 & 450 & 439 & 443\\
\hline
\end{tabular} \phantom{\begin{array}{ll}
 	 \textrm{In Wasser}: \\ \textrm{werte in mm} \\ \textrm{Unsicherheit: $\pm 0.1$mm}
 \end{array}
}$$

F\"ur die W\"armekapazit\"at gilt:\\

\begin{equation}
C=C_{Kal}+C_T+m_wc_w,\label{eq1}
\end{equation}

was umgestellt werden kann zu:

\begin{equation}
c_w=\frac{C-C_{Kal}-C_T}{m_w}.\label{eq2}
\end{equation}

$C$ wird hier mittels der folgenden Gleichungen bestimmt:

\begin{equation}
C=\frac{Q}{\Delta T}
\end{equation}

\begin{equation}
W_R=mgn\pi d=Q
\end{equation}

Mit unseren Messwerten und dem $uncertainties$ Paket in Python berechnen wir damit:\\
$$\begin{tabular}{|c|c|c|c|}
\hline
\textrm{Messung} & 1 & 2 & \textrm{Mittelwert}\\
\hline
\textrm{W\"armekapazit\"at} $\left[\mathrm{\nicefrac{\mathrm{J}}{K}}\right]$ & $8300\pm1400$ & $3500\pm600$ & $5900\pm800$\\
\hline
\end{tabular}$$

Diese Rechnungen wurden mit dem \textit{uncertainties} Paket in Python durchgef\"uhrt. Siehe Abbildung (\ref{ab1}).\\

Der Fehler der Messung mit dem Sch\"urholz Apparat ist aufgrund der gro\ss en Ungenauigkeit der Temperatur und der Anzahl Drehungen sehr gro\ss. Au\ss erdem ist es recht wahrscheinlich, dass $F_{R}$ und $F_{G}$ sich nicht st\"andig ganz ausgleichen, also dadurch auch eine Unsicherheit entsteht. Der Literaturwert hierzu ist $4185.5$ $\mathrm{\nicefrac{\mathrm{J}}{K}}$. Der Unterschied ist aufgrund der Messungenauigkeiten und niedrigen Anzahl Messungen sehr gro\ss.


\subsubsection{Unsicherheitsvergleich mit Streuung}

Aus den 5 bzw. 4 Einzelwerten der beiden Messungen ergeben sich folgende Streuungen:\\

$$s_\rho=279,\ \mathrm{\nicefrac{kg}{m^3}},\ s_{\rho_{Fl}}=49.1,\ \mathrm{\nicefrac{kg}{m^3}}$$

Diese Werte sind erheblich kleiner als erwartet. Der Grund daf\"ur k\"onnte bei einer zu groben Absch\"atzung der Messungenauigkeit oder aufgrund der geringen Anzahl an Einzelmessungen (5 bzw. 4) liegen.\\

Geht man von einer halb so gro\ss en Messungenauigkeit aus, so erh\"alt man Fehlerabsch\"atzungen von etwa 2000 $\mathrm{\nicefrac{kg}{m^3}}$, was immer noch nicht konsistent mit der Absch\"atzung mittels $s_\rho$ ist. Daher m\"ussen wir davon ausgehen, dass die kleinen Werte von $s_\rho\textrm{ und}\ s_{\rho_{Fl}}$ durch die geringe Anzahl an Einzelmessungen zustande gekommen sind.\\

Aus den Unsicherheiten der Einzelmessungen ergeben sich Standardabweichungen des Mittelwerts von:\\
\[
s_{\overline{\rho}}=(3987\pm125)\nicefrac{kg}{m^3},\ 
s_{\overline{\rho_{Fl}}}=(505\pm25)\nicefrac{kg}{m^3}
\]

Bei der Messreihe mit der unbekannten Fl\"ussigkeit gibt es einen systematischen Fehler aufgrund der verwendeten gemessenen Dichte des K\"orpers, die unter umst\"anden zu gro\ss\ oder zu klein gesch\"atzt wurde und als Referenz dient.

\subsection{Teil B - Oberfl\"achenspannung}

Die Oberfl\"achenspannung von Wasser und Ethanol wurde mit der Abrei\ss methode gemessen. F\"ur diese gilt:

\begin{equation}
\sigma=\frac{F_(s_{max})}{2l}\label{eqo}
\end{equation}

$$\begin{tabular}{|c|c|c|c|c|c|c|c|}
\hline
\textrm{H\"ohe }$[mm]$ & 2 & 4 & 5 & 8 & 9 & 10\\
\hline
$\textrm{Kraft }[mN]$ & $0.26\pm0.03$ & $0.33\pm0.03$ & $1.25\pm0.03$ & $3.9\pm0.03$ & $4.75\pm0.03$\\
\hline
\end{tabular} $$\ref{tabow}

Die L\"ange $l$ des Drahts betr\"agt WERT EINHEIT. Die Messwerte befinden sich in Tabelle (\ref{tabow}) f\"ur Wasser und (\ref{taboe}) f\"ur Ethanol. Deren Graphische Darstellungen in den Graphiken (\ref{ab3}) und (\ref{ab4}). Die Sigmoidfunktion $$d\times\frac{1}{1+\exp(-c\times(x-a))}+b$$ wurde mit der curve\_fit Funktion von Python an die Messpunkte angepasst. Aus den Graphiken lesen wir folgende Werte f\"ur $F_{s_{max}}$ ab:

TABELLE
%5.16, 4.75, 

Die Fehler wurden durch Streuung der Messpunkte um die angepassten Kurven abgesch\"atzt.

Mit der Formel (\ref{eqo}) ergibt sich f\"ur die Oberfl\"achenspannung 

TABELLE

MITTELWERT


% Anhang

\section{Anhang: Tabellen und Diagramme}

Tabelle  wurde mit der Umgebung "`tabular"' erzeugt
und mit der Umgebung "`table"' eingebunden.  



\end{document}