\documentclass[11pt,a4paper]{article}

\usepackage[utf8]{inputenc} 
\usepackage[T1]{fontenc} 
\usepackage{lmodern}
\usepackage[margin=2cm]{geometry}
\usepackage[german]{babel}
\usepackage{amsmath} 
\usepackage{graphicx} 
\usepackage{booktabs}
\usepackage{hyperref}
\hypersetup{
    colorlinks,
    citecolor=red,
    filecolor=black,
    linkcolor=black!20!blue!90!,
    urlcolor=black} 
\usepackage{nicefrac}
\usepackage[table]{xcolor}
\usepackage{tocloft}
\usepackage{array}

\newcolumntype{L}[1]{>{\centering\let\newline\\\arraybackslash\hspace{0pt}}m{#1}}

\setlength{\parindent}{0pt}
\setlength{\parskip}{1ex plus 0.5ex minus 0.5ex}

\definecolor{incolor}{rgb}{0.0, 0.0, 0.5}

\hbadness=99999

\newcommand{\refpy}[1]{Siehe Anhang: \textit{Rechnungen in Python} (\texttt{{\color{incolor}In [{\color{incolor}#1}]}})}
\newcommand{\dif}{\mathop{}\!\mathrm{d}}
\newcommand{\vphi}{\varphi}
\newcommand{\halftime}[4]{\begin{figure}[h]
\begin{minipage}{.#1\textwidth}#3\end{minipage}\begin{minipage}{.#2\textwidth}
\centering
#4\end{minipage}
\end{figure}}


\begin{document}


{
\centering 
\large 
Physiklabor für Anf\"anger*innen \\
Ferienpraktikum im Sommersemester 2018 \\[4mm]
\textbf{\LARGE 
Versuch 22: Kreisel
} \\[3mm]
(durchgef\"uhrt am 21.09.2018 bei Adrian Hauber) \\
Andréz Gockel, Patrick M\"unnich\\
\today \\[10mm]
}

\vspace{50pt}
\tableofcontents
\vspace{22pt}
\listoftables
\vspace{22pt}
\listoffigures
\pagebreak

\section{Ziel des Versuchs}

Dieser Versuch dient dazu, freie Rotation eines Systems mit von au\ss en wirkender Drehmomente darzustellen. Insbesondere wird hier auf Pr\"azession eingegangen. Dazu Nutzt man einen Kreisel, an dem Pr\"azession und Rotation betrachtet werden.

\section{Versuch}

\subsection{Theorie}

Starre K\"orper haben Tr\"agheitsmomente in Form eines Tensors. Bestimmte Achsen sind jedoch vielsagend. Diese formen einen Tr\"agheitsellipsoid mit den Hauptachsen $I_A$, $I_B$ und $I_C$. Bei symmetrischen Kreisel, wie hier vorhanden, sind $I_B$ und $I_C$ gleich. $I_A$ ist das Tr\"agheitsmoment bez\"uglich der Figurenachse. Bei Kreiseln ist dieses gleichzeitig unsere gr\"o\ss te Komponente beim Tr\"agheitsellipsoid.

Tr\"agheitsmomente sind allgemein mit der Drehgeschwindigkeit $\vec{\omega}$ und dem Drehimpuls $\vec{L}$ verbunden:

\begin{equation}
\vec{L}=I\vec{\omega}
\end{equation}

Sto\ss t man einen an der Figurenachse gebundenen Kreisel an, so wirkt ein Drehmoment:

\begin{equation}
\vec{M}=\vec{r}\times\vec{G}=\frac{\dif\vec{L}}{\dif t}\label{m=rg}
\end{equation}

$\vec{r}$ bezeichnet hier den Vektor vom Unterst\"utzungspunkt zum Schwerpunkt ud $\vec{G}$ die Gewichtskraft im Schwerpunkt.

Da dies sich nur in der senkrechten Komponente \"andert, \"andert sich der Betrag des Drehimpulses nicht. Die einzige \"Anderung ist die Richtung, welche zu einer Drehbewegung in der Senkrechte f\"uhrt. Diese Bewegung wird als Pr\"azession bezeichnet, mit der Winkelgeschwindigkeit $\vec{\omega}_P$. Zusammen mit der Drehgeschwindigkeit $\vec{\omega}_F$ um die Figurenachse ergibt sich die momentane Drehgeschwindigkeit $\vec{\omega}$.

Bei einem pr\"azedierenden Kreisel gibt es also eine Kreisbewegung mit Radius $L\sin\phi$ mit $\phi$ als Winkel zwischen $\vec{L}$ und der vertikalen Achse. Die Rotationsgeschwindigkeit hat einen Betrag von $\omega_P=\nicefrac{\dif\vphi}{\dif t}$, wobei $\vphi$ der azimutale Winkel ist. Die \"Aenderung des Drehimpulses ist also eine Richtungs\"anderung $\dif\vphi=\frac{\dif L}{L\sin\phi}$. Mit diesem Wissen kommen wir also zu

\[
\omega_P=\frac{\dif\phi}{\dif t}=\frac{\dif L}{\dif tL\sin\phi}
\]

Dies k\"onnen wir umschreiben zu

\begin{equation}
\frac{\dif\vec{L}}{\dif t}=\vec{\omega}_P\times\vec{L}.
\end{equation}

Mit (\ref{m=rg}) kommen wir zu der Gleichung

\begin{equation}
\vec{r}\times\vec{G}=\vec{\omega}_P\times\vec{L}\label{rgisoml}
\end{equation}

Da wir in diesem Versuch mit einem schnell rotierenden Kreisel arbeiten, also $\omega_F>>\omega_P$ gilt n\"aherungsweise $\sin(\vec{r},\vec{G})=\sin(\vec{\omega}_P,\vec{L})$. Als n\"achstes schreiben wir (\ref{rgisoml}) f\"ur die Betr\"age der Vektoren um:

\begin{equation}
rG\sin(\vec{r},\vec{G})=\omega_PL\sin(\vec{\omega}_P,\vec{L}).
\end{equation}

Nutzen wir unsere N\"aherung aus, so k\"onnen wir schreiben

\begin{equation}
rG\approx\omega_AL\approx\omega_FL.
\end{equation}

Hier ist $\omega_A$ die Drehgeschwindigkeitskomponente, die mit $\omega_B$ entlang zweier Haupttr\"agheitsachsen des Kreisels verl\"auft und sich mit $I_A$ und $I_B$ zusammen zu $\vec{L}$ vektoriell addieren lassen. 

Mit $L\approx I\omega_A\approx I\omega_F$ k\"onnen wir sagen, dass $L\approx\omega_P\omega_AI\approx I\omega_P\omega_F$. Damit kommen wir auf unsere gesuchte Gleichung

\begin{equation}
I_A=\frac{rG}{\omega_F\omega_P},\label{eqIA}
\end{equation} 

welche wir f\"ur unsere Rechnungen ben\"otigen.

\subsection{Aufbau}


\halftime{5}{5}{Es wurde ein Kreiselrad mit verstellbarer Kreiselachse verwendet. Dieser wurde auf ein Stativ mit einer drehbaren Halterung die es ermöglichte eine Präzessionsbewegung zu durchlaufen. Eine Federwaage wurde verwendet um die Masse des Kreisels zu bestimmen. Es wurden zwei Stoppuhren benutzt um jeweils die Präzessionsfrequenz und die Rotationsfrequenz zu messen. 
}{\fbox{\includegraphics[width=0.9\textwidth]{Krpr}}
   \renewcommand\thefigure{B1}
\caption[Präzessierender Kreisel]{Präzessierender Kreisel \cite{Anleitung}}
\label{Pic:1}}

\subsection{Durchführung}

Zuerst wurde Masse des Kreisels gemessen. Zunächst wurde die Kreiselachse eingestellt, dann wurde das Kreiselrad angehoben und per hand im Uhrzeigersinn gedreht. Der rotierende Kreisel wurde dann vorsichtig auf die Halterung platziert und gekippt. Dann wurden 10 Rotationen zeitlich gemessen, und eine Präzessionsbewegung. Dies wurde für 10 verschiedene Kreiselachsen vier mal durchgeführt. Die Drehrichtung der Präzessionsbewegung wurde jedesmal notiert. Es konnten nur die Drehrichtung bestimmt werden für die Einstellungen wo der Schwerpunkt zu nahe dem Unterstützungspunkt war.

\subsection{Auswertung}

Erstmals ist es wichtig zu wissen, dass unser gesuchtes $I_A$ eine Konstante ist. Wollen wir diese mit (\ref{eqIA}) finden, so brauchen wir $r$, $m$, $g$, $\omega_F$ und $\omega_P$. Die Masse $m$ k\"onnen wir leicht messen und wir bekommmen als Ergebnis $(4.52\pm0.020$)\,kg. F\"ur $g$ verwenden wir einfach $9.81\,\nicefrac{\mathrm{m}}{\mathrm{s}^2}$. $\omega_F$ und $\omega_P$ k\"onnen wir \"uber die Periodendauer T berechnen: 
\[
\omega=\frac{2\pi}{T}.
\]

Uns fehlt also nur noch $r$. Da $I_A$ eine Konstante ist, k\"onnen wir aus der Gleichung (\ref{eqIA}) schlie\ss en:
\begin{equation}
r=\frac{\omega_F\omega_P}{G}\label{eqgraph}
\end{equation}
F\"ur die rechte Seite der Gleichung ist alles gemessen. Wir k\"onnen dies in eine Graphik auftragen, in der wir dann einen linearen Zusammenhang sehen werden. Wir folgern, dass $r$ wohl aus unserem eingestellten Abstand $x$ summiert mit einem Offset $x_0$ besteht.

Nutzen wir f\"ur als lineare Gleichung f\"ur eine Ausgleichsgrade von dne Messpunkten
\[
y=ax+b,
\]
so k\"onnen wir unser $x_0$ als $-\frac{b}{a}$ festlegen.

THIS ISN'T DONE



\section{Diskussion}

XXXX

\pagebreak

\section{Anhang: Tabellen und Diagramme}

\begin{table}[h]
\centering
\caption{Messwerte} \vspace{11pt}
$\begin{array}{l}
\textrm{Unsicherheiten:}\\
\textrm{Zeit: } \pm 0.3 \textrm{s}\\
\textrm{Länge: } \pm 0.05 \textrm{cm}\\
\end{array}$
\begin{tabular}{ r L{3cm} L{3cm} }
\toprule
$l$\textrm{ in cm} & \textrm{Präzession umlaufdauer\textrm{ in s}} & \textrm{10 Rotationen umlaufdauer}\textrm{ in s} \\
\midrule
1 & 6.6 & 5.1\\
1 & 4.7 & 6.9\\
1 & 3.6 & 8.1\\
1 & 7.9 & 4.2\\
\hline
2 & 6.5 & 6.7\\
2 & 6.7 & 6.7\\
2 & 7.0 & 6.0\\
2 & 9.4 & 5.5\\
\hline
3 & 13.7 & 6.8\\
3 & 14.4 & 6.2\\
3 & \phantom{0}9.8 & 7.8\\
3 & 13.4 & 6.1\\
\hline
8 & 5.7 & 6.4\\
8 & 6.6 & 5.5\\
8 & 5.6 & 6.6\\
8 & 6.4 & 6.0\\
\hline
9 & 5.1 & 5.7\\
9 & 4.7 & 6.5\\
9 & 4.1 & 8.0\\
9 & 6.3 & 4.5\\
\hline
10 & 1.7 & 6.7\\
10 & 2.7 & 5.0\\
10 & 3.9 & 3.6\\
10 & 3.2 & 4.0\\
\bottomrule
\end{tabular}
\phantom{$\begin{array}{l}
\textrm{Unsicherheiten:}\\
\textrm{XXXX: } \pm XX \textrm{XX}\\
\end{array}$}
\label{Tab:X}
\end{table}

%\begin{figure}[p]
%\centering
%\fbox{\includegraphics[width=0.8\textwidth]{NAME}}
%\renewcommand\thefigure{BX}
%\caption[XXXX]{XXXX}
%\label{Abb:X}
%\end{figure}

\begin{thebibliography}{9}
\bibitem{Uncertainties}''Correlations between variables are automatically handled, which sets this module apart from many existing error propagation codes.'' - https://pythonhosted.org/uncertainties/
\bibitem{Anleitung} Physikalisches Institut der Albert-Ludwigs-Universität Freiburg (Hrsg.) (08/2018): Versuchsanleitungen zum Physiklabor für Anfänger*innen, Teil 1, Ferienpraktikum im Sommersemester 2018.
\end{thebibliography}

\end{document}