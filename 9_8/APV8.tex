\documentclass[11pt,a4paper]{article}

\usepackage[utf8]{inputenc} 
\usepackage[T1]{fontenc} 
\usepackage{lmodern}
\usepackage[margin=2cm]{geometry}
\usepackage[german]{babel}
\usepackage{amsmath} 
\usepackage{graphicx} 
\usepackage{booktabs}
\usepackage{hyperref}
\hypersetup{
    colorlinks,
    citecolor=red,
    filecolor=black,
    linkcolor=black!20!blue!90!,
    urlcolor=black} 
\usepackage{nicefrac}
\usepackage[table]{xcolor}
\usepackage{tocloft}

\setlength{\parindent}{0pt}
\setlength{\parskip}{1ex plus 0.5ex minus 0.5ex}

\definecolor{incolor}{rgb}{0.0, 0.0, 0.5}

\hbadness=99999

\newcommand{\refpy}[1]{Siehe Anhang: \textit{Rechnungen in Python} (\texttt{{\color{incolor}In [{\color{incolor}#1}]}})}
\newcommand\dif{\mathop{}\!\mathrm{d}}
\newcommand{\halftime}[4]{\begin{figure}[h]
\begin{minipage}{.#1\textwidth}#3\end{minipage}\begin{minipage}{.#2\textwidth}
\centering
#4\end{minipage}
\end{figure}}
\renewcommand{\vec}{\boldsymbol}

\begin{document}

% name of experiment
% date of experiment
% name of assistant
{
\centering 
\large 
Physiklabor für Anf\"anger*innen \\
Ferienpraktikum im Sommersemester 2018 \\[4mm]
\textbf{\LARGE 
Versuch 8: Viskosit\"at aus dem Durchstr\"omen einer Kapillare
} \\[3mm]
(durchgef\"uhrt am 26.09.2018 bei Pascal Wunderlin) \\
Andréz Gockel, Patrick M\"unnich\\
\today \\[10mm]
}

\vspace{50pt}
\tableofcontents
\vspace{22pt}
\listoftables
\vspace{22pt}
\listoffigures
\pagebreak

\section{Ziel des Versuchs}

Das Ziel des Versuchs ist es, den Zusammenhang zwischen Str\"omungsgeschwindigkeit, Viskosit\"at, Druckdifferenz und geometrischen Parametern darzustellen. Hierzu wird erstmal das Gesetz von Hagen-Poiseuille durch Messung der Volumenstromst\"arke durch verschiedene Kapillare \"uberpr\"uft, und dann die Viskosit\"at von Wasser bestimmt.

\section{Teil 1}

\subsection{Theorie}

Ist eine Laminarstr\"omung vorhanden, also sind keine Turbulenzen zwischen den einzelnen infitesimalen Wasserschichten vorhanden, so gilt f\"ur die Volumenstromst\"arke $I_V$ das Gesetz von Hagen-Poiseuille:

\begin{equation}
I_V=\frac{V}{t}=\frac{\pi R^4\Delta p}{8\eta l}\label{hagen}
\end{equation}

Zur Herleitung dessen wird die Definition der Viskosit\"at genutzt:

\begin{equation}
F=\eta A\frac{\dif v}{\dif x}
\end{equation}

Um die Druckdifferenz $\Delta p$ zu berechnen, benötigt man die Steighöhe $h$ und die Dichte von Wasser $\rho_w$.
Aus
$$F_G = mg = \rho_w Vg = \rho_w Ahg = F_2$$
Wobei $V$ das Volumen des Wassers im Steigrohr mit $A$ die Querfläche des Steigrohrs mal $h$ ist.
Mit $p = \frac{(F2-F1)}{A}$ und $F_1 = 0$, da der Aussendruck durch das Loch ausgeglichen wird, bekommt man für $\Delta p$:
$$\Delta p = \frac{\rho_w Ahg}{A} = \rho_w h \vec{g}$$


\subsection{Aufbau}

% describe set up
% insert pic name, designation, toc caption, caption, label
%\halftime{5}{5}{TEXT}{\fbox{\includegraphics[width=0.5\textwidth]{NAME}}
%   \renewcommand\thefigure{BX}
%\caption[XXXX]{XXXX \cite{Anleitung}}
%\label{Pic:X}}

\subsection{Durchführung}

% describe exp.
XXXX

\subsection{Auswertung}

In der Formel $I_V=\frac{V}{t}$ haben sowohl $V$ und $t$ Fehler. Wir verwenden hier also die verallgemeinerte Formel f\"ur Quotienten:
$$
\left\vert\frac{\Delta z}{z}\right\vert=\sqrt{\left(a\frac{\Delta x}{x}\right)^2+\left(b\frac{\Delta y}{y}\right)^2+\ldots}\textrm{ f\"ur }z=x^a\ y^b\ldots
$$
Hier also:
$$
\left\vert\frac{\Delta I_V}{I_V}\right\vert=\sqrt{\left(\frac{\Delta V}{V}\right)^2+\left(-1\frac{\Delta t}{t}\right)^2}
$$
Da $\frac{d^4}{l}$ aus Werten ohne vorhandenem Fehler bestehen, berechnen wir daf\"ur keinen Fehler.

Um unseren Mittelwert zu berechnen, rechnen wir ganz leicht mit 
$$
\frac{\sum_{i=1}^n I_{V_i}}{n}
$$
den Nominalwert, und mit
$$
s_x=\sqrt{\frac{1}{n-1}\sum_{i=1}^n(x_i-\overline{x})^2}
$$
die Standardunsichertheit dessen.

Unsere Mittelwerte der $I_V$ f\"ur jede Position werden dann gegen $\nicefrac{d^4}{l}$ aufgetragen, siehe Abbildung (\ref{Abb1}).

Da wir jedoch klar erkennen k\"onnen, dass $I_{V_4}$ mit der linearen Steigung der anderen Werte nicht \"ubereinstimmt, lassen wir diesen Wert weg und erhalten die Gerade, welche in Abbildung (\ref{Abb2}) gefunden werden kann.

In der Formel $I_V=\frac{V}{t}$ haben sowohl $V$ und $t$ Fehler. Wir verwenden hier also die verallgemeinerte Formel f\"ur Quotienten:
$$
\left\vert\frac{\Delta z}{z}\right\vert=\sqrt{\left(a\frac{\Delta x}{x}\right)^2+\left(b\frac{\Delta y}{y}\right)^2+\ldots}\textrm{ f\"ur }z=x^a\ y^b\ldots
$$
Hier also:
$$
\left\vert\frac{\Delta I_V}{I_V}\right\vert=\sqrt{\left(\frac{\Delta V}{V}\right)^2+\left(-1\frac{\Delta t}{t}\right)^2}
$$
Da $\frac{d^4}{l}$ aus Werten ohne vorhandenem Fehler bestehen, berechnen wir daf\"ur keinen Fehler.

Um unseren Mittelwert zu berechnen, rechnen wir ganz leicht mit 
$$
\frac{\sum_{i=1}^n I_{V_i}}{n}
$$
den Nominalwert, und mit
$$
s_x=\sqrt{\frac{1}{n-1}\sum_{i=1}^n(x_i-\overline{x})^2}
$$
die Standardunsichertheit dessen.

Unsere Mittelwerte der $I_V$ f\"ur jede Position werden dann gegen $\nicefrac{d^4}{l}$ aufgetragen, siehe Abbildung (\ref{Abb1}).

Da wir jedoch klar erkennen k\"onnen, dass $I_{V_4}$ mit der linearen Steigung der anderen Werte nicht \"ubereinstimmt, lassen wir diesen Wert weg und erhalten die Gerade, welche in Abbildung (\ref{Abb2}) gefunden werden kann.

Wir erhalten als Ergebnis daraus f\"ur unser $a$ einen Wert von $(0.014\pm0.011)\,\mathrm{mm}^3$

Um aus unseren Werten $\Delta p$ zu berechnen, verwenden wir
$$
\Delta p=\rho_w hg
$$

Da der einzige Wert mit einem Fehler $h$ ist, rechnen wir einfach mit
$$
\Delta z=\left|{\dif f}{\dif x}\right|\Delta x\textrm{ f\"ur }z=f(x)
$$
unseren Fehler aus.
Mit $\rho_w=1000\,\frac{\mathrm{kg}}{\mathrm{m}^3}$, $g=9.81\,\frac{\mathrm{m}}{\mathrm{s}^2}$ und $h=(135\pm3)\,\mathrm{mm}$ erhalten wir als Wert $\Delta p=(132\pm29)\,$bar.

Da wir als Endergebnis $\eta$ wollen, m\"ussen wir erstmal die Gleichung (\ref{hagen}) umstellen und wir erhalten:
$$
\eta=\frac{\pi R^4\Delta p}{8I_V l}.
$$
Hier haben $\Delta p$ und $I_V$ Fehler. Wir wenden also wieder die Gleichung f\"ur Produkte an und erhalten:
$$
\left\vert\frac{\Delta\eta}{\eta}\right\vert=\sqrt{\left(\frac{\Delta\Delta p}{\Delta p}\right)^2+\left(-1\frac{\Delta I_V}{I_V}\right)^2}
$$

Als Ergebnis f\"ur $\eta$ erhalten wir f\"ur unsere vier verwendeten Messreihen:
\begin{itemize}
\item $(0.076\pm0.017)\,\mathrm{\frac{kg}{ms}}$
\item $(0.118\pm0.027)\,\mathrm{\frac{kg}{ms}}$
\item $(0.088\pm0.020)\,\mathrm{\frac{kg}{ms}}$
\item $(0.091\pm0.020)\,\mathrm{\frac{kg}{ms}}$
\end{itemize}


Als n\"achstes betrachten wir den durchschnittlichen Fehler der Messungen und die Streuung:

Wir erhalten als durchschnittlichen Fehler $0.006\,$A und als Streuung $0.570\,$A.

\section{Diskussion}

XXXX

\pagebreak

\section{Anhang: Tabellen und Diagramme}

\begin{table}[h]
\centering
\caption{XXXX} \vspace{11pt}
$\begin{array}{l}
\textrm{Unsicherheiten:}\\
\textrm{XXXX: } \pm XX \textrm{XX}\\
\end{array}$
\begin{tabular}{ccc}
\toprule
\textrm{XXXX}/\textrm{XX} & \textrm{XXXX}/\textrm{XX} & \textrm{XXXX}/\textrm{XX} \\
\midrule 
2 & 0.26 & 0.23\\
\hline
4 & 0.33 & 0.25\\
\hline 
5 & & 0.3\\
\hline 
6 & 1.25 & 0.83\\
\hline 
8 & 3.9 & 0.83\\ 
\hline
9 & 4.75 & 4.6\\ 
\hline
10 & 4.7 &\\ 
\bottomrule
\end{tabular}
\phantom{$\begin{array}{l}
\textrm{Unsicherheiten:}\\
\textrm{XXXX: } \pm XX \textrm{XX}\\
\end{array}$}
\label{Tab:X}
\end{table}

%\begin{figure}[p]
%\centering
%\fbox{\includegraphics[width=0.8\textwidth]{NAME}}
%\renewcommand\thefigure{BX}
%\caption[XXXX]{XXXX}
%\label{Abb:X}
%\end{figure}

\begin{thebibliography}{9}
\bibitem{Uncertainties}''Correlations between variables are automatically handled, which sets this module apart from many existing error propagation codes.'' - https://pythonhosted.org/uncertainties/
\bibitem{Anleitung} Physikalisches Institut der Albert-Ludwigs-Universität Freiburg (Hrsg.) (08/2018): Versuchsanleitungen zum Physiklabor für Anfänger*innen, Teil 1, Ferienpraktikum im Sommersemester 2018.
\end{thebibliography}

\end{document}