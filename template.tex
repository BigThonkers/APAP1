\documentclass[11pt,a4paper]{article}

\usepackage[utf8]{inputenc} 
\usepackage[T1]{fontenc} 
\usepackage{lmodern}
\usepackage[margin=2cm]{geometry}
\usepackage[german]{babel}

\setlength{\parindent}{0pt}
\setlength{\parskip}{1ex plus 0.5ex minus 0.5ex}

\usepackage{amsmath} 
\usepackage{graphicx} 
\usepackage{booktabs}
\usepackage[colorlinks]{hyperref}

\begin{document}

{
\centering 
\large 
Physiklabor für Anfänger*innen \\
Ferienpraktikum im Sommersemester 2018 \\[4mm]
\textbf{\LARGE 
Versuch 04: Dichte und Oberflächenspannung
} \\[3mm]
(durchgeführt am 07.09.2018 bei Daniel Bartle) \\
Andréz Gockel, Patrick Münnich\\
\today \\[10mm]
}

\section{Ziel des Versuchs}

Das Ziel dieser versuche ist vorerst die 
Der erste teil des experiments lässt ein das Archimedische Prinzip veranschaulichen, und ermöglicht es die wissenschaftliche Vorgehensweise in praxis zu erleben. 
Der zweite teil dieses experiments ermöglicht uns die Oberflächenspannung von diversen Flüssigkeiten zu bestimmen mit einem torqueometer und einem Draht.

%%%%% Auswertung und Fehleranalyse

\section{Auswertung und Fehleranalyse}


\subsection{Teil A - Dichte}
Wir nehmen die Messwerte und stecken sie irgendwo hin. 

\subsection{Teil B - Oberflächenspannung}
Lorem ipsum dolor sit amet, consectetur adipisici elit, sed eiusmod \dots

\subsection{Dritter Versuchsteil}
Lorem ipsum dolor sit amet, consectetur adipisici elit, sed eiusmod \dots

\subsubsection{Einzelmessung}
Lorem ipsum dolor sit amet, consectetur adipisici elit, sed eiusmod \dots

\subsubsection{Mehrfachmessung}
\label{mehrfachmessung}
Lorem ipsum dolor sit amet, consectetur adipisici elit, sed eiusmod \dots


%%%%% Diskussion der Ergebnisse

\section{Diskussion der Ergebnisse}
\label{diskussion}

Angabe der Ergebnisse in der Form Bestwert $\pm$ Unsicherheit: 
\[ 
 x = (123{,}45 \pm 0{,}02)\,\text{Einheit}
\]

Angabe der relativen Unsicherheit: 
\[
 \left| \frac{u_x}{x_0} \right| = 0{,}00016 = 1{,}6\times 10^{-4} = 0{,}016\%
\]

Eine kurze Formel $y=x^2$ im Lauftext. 

Eine große abgesetzte Formel mit automatischer Nummerierung: 
\begin{equation}
 \langle y \rangle = \sum_{j=1}^{n}{ \left( \frac{1}{\sqrt{2\pi}} \cdot \int_{-\infty}^{j}{ \sin{\phi} \,\text{d}\phi} \right)^2 } 
\label{formel}
\end{equation}
Mithilfe des Labels kann später auf diese Formel verwiesen werden \eqref{formel}.

Eine Auflistung kann so aussehen: 
\begin{itemize}
	\item erster Punkt
	\item zweiter Punkt
	\item \dots
\end{itemize}
	 
Eine nummerierte Auflistung so: 
\begin{enumerate}
	\item erster Punkt
	\item zweiter Punkt
	\item \dots
\end{enumerate}
	 
	
%%%%% Anhang

\section{Anhang: Tabellen und Diagramme}

Tabelle \ref{tabelle} wurde mit der Umgebung "`tabular"' erzeugt
und mit der Umgebung "`table"' eingebunden.  

\begin{table}[p]
\centering
\begin{tabular*}{0.99\textwidth}{@{\extracolsep{\fill}}cccccc}
\toprule
 $x$ & $u_x$ &  $y$  &  $u_y$ &  $y^2$  &  $2 y u_y$  \\
 m & m &  s & s &  s$^2$ & s$^2$  \\
\midrule
  0,16 & 0,01 & 0,686 & 0,029 & 0,47 & 0,040 \\
  0,33 & 0,01 & 1,046 & 0,031 & 1,09 & 0,065  \\
  0,52 & 0,01 & 1,381 & 0,026 & 1,91 & 0,072  \\
  0,69 & 0,01 & 1,607 & 0,021 & 2,58 & 0,067  \\
  0,85 & 0,01 & 1,785 & 0,022 & 3,19 & 0,079  \\	
\bottomrule
\end{tabular*}
\caption{Eine Tabellenunterschrift.}
\label{tabelle}
\end{table}


%\begin{figure}[p]
%\centering
%\includegraphics[width=0.8\textwidth]{bild.png}
%\caption{Eine Bildunterschrift.}
%\label{abbildung}
%\end{figure}


\end{document}
%%%%%%%%%%%%%%%%%%%%%%%%%%%%%%%%%%%%%%%%%