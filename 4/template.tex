\documentclass[11pt,a4paper]{article}

%\usepackage[utf8]{inputenc} 
%\usepackage[T1]{fontenc} 
%\usepackage{lmodern}
%\usepackage[margin=2cm]{geometry}
%\usepackage[german]{babel}
%\usepackage[table]{xcolor}
\usepackage{nicefrac}

\setlength{\parindent}{0pt}
\setlength{\parskip}{1ex plus 0.5ex minus 0.5ex}

\usepackage{amsmath} 

%\usepackage{graphicx} 
%\usepackage{booktabs}
%\usepackage[colorlinks]{hyperref}

\begin{document}

{
\centering 
\large 
Physiklabor für Anf\"anger*innen \\
Ferienpraktikum im Sommersemester 2018 \\[4mm]
\textbf{\LARGE 
Versuch 04: Dichte und Oberflächenspannung
} \\[3mm]
(durchgef\"uhrt am 07.09.2018 bei Daniel Bartle) \\
Andréz Gockel, Patrick M\"unnich\\
\today \\[10mm]
}

\section{Ziel des Versuchs}

Der Versuch ist in zwei Teile geteilt, welche dazu dienen, grundlegende Eigenschaften von Fl\"ussigkeiten experimentell zu bestimmen. Im Teil A bestimmt man die Dichte von Wasser und einer unbekannten Fl\"ussigkeit mithilfe einer Jollyschen Federwaage. Im Teil B bestimmt man die Oberfl\"achen\-spannung von Wasser durch Messen der Abrisskraft mithilfe eines Torsionskraftmessers.

%%%%% Auswertung und Fehleranalyse

\section{Auswertung und Fehleranalyse}


\subsection{Teil A - Dichte}

\subsubsection{Aufgabenstellung}
Mit Hilfe der Jollyschen Federwaage sind zu bestimmen
\begin{enumerate}
\item{die Dichte eines geometrisch einfach gestalteten K\"orpers, wobei ein Vergleich mit den aus den
geometrischen Abmessungen und dem Gewicht des K\"orpers gewonnenem Wert durchzuf\"uhren
ist,}
\item{die Dichte einer unbekannten Fl\"ussigkeit.}
\end{enumerate}

\subsubsection{Auswertung}

Zur ersten Aufgabe:\\
Die Messungen wurden mit einer Metallkugel durchgef\"uhrt mit einem Durch\-messer von $d=(1.2\pm0.03)$cm und einer Masse von $m=(7.03\pm0.005)$g. Mit der Formel f\"ur das Volumen, $V=\frac{\pi}{6}d^3$, und f\"ur die Dichte, $\rho=\frac{m}{V}$, ergibt sich ein Wert von $(7810\pm550)\nicefrac{\textrm{kg}}{\textrm{m}^3}$. Hierbei wurde der Fehler \"uber die Potenzformel des Gau\ss 'schen Fehlerfortpflanzungsgesetzes bei Vernachl\"assigung des Fehlers der Masse bestimmt.\\

%$$
% \begin{array}{ll}
% 	 \textrm{In Wasser}: \\ \textrm{werte in mm} \\ \textrm{Unsicherheit: $\pm 0.1$mm}
% \end{array}
%\rowcolors{2}{gray!10}{white}
%\noindent%
%\begin{tabular}{|l | ccccc|}
%\hline
%Messung & 1 & 2 & 3 & 4 & 5\\
%\hline
%Ruhelage & 441 & 479 & 473 & 463 & 468\\
%Nicht eingetaucht & 414 & 453 & 447 & 436 & 440\\
%Eingetaucht & 417 & 456 & 450 & 439 & 443\\
%\hline
%\end{tabular} \phantom{\begin{array}{ll}
% 	 \textrm{In Wasser}: \\ \textrm{werte in mm} \\ \textrm{Unsicherheit: $\pm 0.1$mm}
% \end{array}
%}$$

% TABELLE HIER UND $X_1 X_2 X_0$ DEFINIEREN\\

% Mit den Messwerten in \\

INSERT 2.1, 2.2 MIT GROESSENANGABEN

ergibt sich
\begin{equation}
\rho=\rho_{Fl}\frac{x_1-x_0}{x_1-x_2}
\end{equation}
F\"ur unsere Messwerte aus Tabelle 1 erhalten wir

$$\begin{tabular}{|c|c|c|c|c|}
\hline
$8982\pm4260$ & $8649\pm4104$ & $8649\pm4104$ & $8982\pm4260$ & $9314\pm4416$\\
\hline
\end{tabular} \textrm{ in } \nicefrac{\textrm{kg}}{\textrm{m}^3}$$

FEHLERDISKUSSION

\subsection{Teil B - Oberfl\"achenspannung}
Lorem ipsum dolor sit amet, consectetur adipisici elit, sed eiusmod \dots

\subsection{Dritter Versuchsteil}
Lorem ipsum dolor sit amet, consectetur adipisici elit, sed eiusmod \dots

\subsubsection{Einzelmessung}
Lorem ipsum dolor sit amet, consectetur adipisici elit, sed eiusmod \dots

\subsubsection{Mehrfachmessung}
\label{mehrfachmessung}
Lorem ipsum dolor sit amet, consectetur adipisici elit, sed eiusmod \dots


%%%%% Diskussion der Ergebnisse

\section{Diskussion der Ergebnisse}
\label{diskussion}

Angabe der Ergebnisse in der Form Bestwert $\pm$ Unsicherheit: 
\[ 
 x = (123{,}45 \pm 0{,}02)\,\text{Einheit}
\]

Angabe der relativen Unsicherheit: 
\[
 \left| \frac{u_x}{x_0} \right| = 0{,}00016 = 1{,}6\times 10^{-4} = 0{,}016\%
\]

Eine kurze Formel $y=x^2$ im Lauftext. 

Eine große abgesetzte Formel mit automatischer Nummerierung: 
\begin{equation}
 \langle y \rangle = \sum_{j=1}^{n}{ \left( \frac{1}{\sqrt{2\pi}} \cdot \int_{-\infty}^{j}{ \sin{\phi} \,\text{d}\phi} \right)^2 } 
\label{formel}
\end{equation}
Mithilfe des Labels kann später auf diese Formel verwiesen werden \eqref{formel}.

Eine Auflistung kann so aussehen: 
\begin{itemize}
	\item erster Punkt
	\item zweiter Punkt
	\item \dots
\end{itemize}
	 
Eine nummerierte Auflistung so: 
\begin{enumerate}
	\item erster Punkt
	\item zweiter Punkt
	\item \dots
\end{enumerate}
	 
	
%%%%% Anhang

\section{Anhang: Tabellen und Diagramme}

Tabelle \ref{tabelle} wurde mit der Umgebung "`tabular"' erzeugt
und mit der Umgebung "`table"' eingebunden.  

%\begin{table}[p]
%\centering
%\begin{tabular*}{0.99\textwidth}{@{\extracolsep{\fill}}cccccc}
%\toprule
% $x$ & $u_x$ &  $y$  &  $u_y$ &  $y^2$  &  $2 y u_y$  \\
% m & m &  s & s &  s$^2$ & s$^2$  \\
%\midrule
%  0,16 & 0,01 & 0,686 & 0,029 & 0,47 & 0,040 \\
%  0,33 & 0,01 & 1,046 & 0,031 & 1,09 & 0,065  \\
%  0,52 & 0,01 & 1,381 & 0,026 & 1,91 & 0,072  \\
%  0,69 & 0,01 & 1,607 & 0,021 & 2,58 & 0,067  \\
%  0,85 & 0,01 & 1,785 & 0,022 & 3,19 & 0,079  \\	
%\bottomrule
%\end{tabular*}
%\caption{Eine Tabellenunterschrift.}
%\label{tabelle}
%\end{table}


%\begin{figure}[p]
%\centering
%\includegraphics[width=0.8\textwidth]{bild.png}
%\caption{Eine Bildunterschrift.}
%\label{abbildung}
%\end{figure}


\end{document}
%%%%%%%%%%%%%%%%%%%%%%%%%%%%%%%%%%%%%%%%%