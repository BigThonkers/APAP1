\documentclass[11pt,a4paper]{article}

\usepackage[utf8]{inputenc} 
\usepackage[T1]{fontenc} 
\usepackage{lmodern}
\usepackage[margin=2cm]{geometry}
\usepackage[german]{babel}
\usepackage{amsmath} 
\usepackage{graphicx} 
\usepackage{booktabs}
\usepackage{hyperref}
\usepackage{tikz}
\hypersetup{
    colorlinks,
    citecolor=red,
    filecolor=black,
    linkcolor=black!20!blue!90!,
    urlcolor=black} 
\usepackage{nicefrac}
%\usepackage[table]{xcolor}
\usepackage{tocloft}

\setlength{\parindent}{0pt}
\setlength{\parskip}{1ex plus 0.5ex minus 0.5ex}

\definecolor{incolor}{rgb}{0.0, 0.0, 0.5}

\hbadness=99999

\newcommand{\refpy}[1]{Siehe Anhang: \textit{Rechnungen in Python} (\texttt{{\color{incolor}In [{\color{incolor}#1}]}})}
\newcommand\dif{\mathop{}\!\mathrm{d}}
\newcommand{\halftime}[4]{\begin{figure}[h]
\begin{minipage}{.#1\textwidth}#3\end{minipage}\begin{minipage}{.#2\textwidth}
\centering
#4\end{minipage}
\end{figure}}
\renewcommand{\vec}{\boldsymbol}
\newcommand\mean{\begin{equation}
\frac{\sum_{i=1}^n x_{i}}{n}\label{mean}
\end{equation}}
\newcommand\meanstd{\begin{equation}
s_x=\sqrt{\frac{1}{n-1}\sum_{i=1}^n(x_i-\overline{x})^2}\label{meanstd}
\end{equation}}
\newcommand\prodquo{\begin{equation}\left\vert\frac{\Delta z}{z}\right\vert=\sqrt{\left(a\frac{\Delta x}{x}\right)^2+\left(b\frac{\Delta y}{y}\right)^2+\ldots}\textrm{ f\"ur }z=x^a\ y^b\ldots\end{equation}}
\newcommand\tfunc{\begin{equation}
\frac{\vert x-y_0\vert}{u_x}
\end{equation}}

\begin{document}

% name of experiment
% date of experiment
% name of assistant
{
\centering 
\large 
Physiklabor für Anf\"anger*innen \\
Ferienpraktikum im Sommersemester 2018 \\[4mm]
\textbf{\LARGE 
Versuch 70: Linsen und Linsensysteme
} \\[3mm]
(durchgef\"uhrt am 28.09.2018 bei Daniel Bartel) \\
Andréz Gockel, Patrick M\"unnich\\
\today \\[10mm]
}

\vspace{50pt}
\tableofcontents
\vspace{22pt}
\listoftables
\vspace{22pt}
\listoffigures
\pagebreak

\section{Ziel des Versuchs}
XXXX
\section{Teil 1}

\subsection{Theorie}

XXXX

\subsection{Aufbau}

% describe set up
% insert pic name, designation, toc caption, caption, label

%\halftime{5}{5}{TEXT}{\fbox{\includegraphics[width=0.5\textwidth]{NAME}}
%   \renewcommand\thefigure{BX}
%\caption[XXXX]{XXXX \cite{Anleitung}}
%\label{Pic:X}}

\subsection{Durchführung}

% describe exp.
XXXX

\subsection{Auswertung}

In diesem Teil wollen wir einfach $\nicefrac{1}{b}$ gegen $\nicefrac{1}{g}$ auftragen. Die gesch\"atzten Fehler werden als Fehlerbalken eingezeichnet. Zum Vergleich werden noch Geraden addiert, welche f\"ur die Linse mit $f=80\,$mm mit
\[
\frac{g}{f}
\]
berechnet wurde und f\"ur die Linsensysteme mit jeweils $f_1=80\,$mm und $f_2=150\,$mm bzw. $f_1=80\,$mm und $f_2=200\,$mm mit
\[
\frac{1}{f_1}+\frac{1}{f_2}-\frac{1}{g}
\]
bestimmt. Die resultierende Graphik kann im Anhang als Abbildung \ref{Abb:1} gefunden werden.

\section{Teil 2}

\subsection{Theorie}

XXXX

\subsection{Aufbau}

% describe set up
% insert pic name, designation, toc caption, caption, label

%\halftime{5}{5}{TEXT}{\fbox{\includegraphics[width=0.5\textwidth]{NAME}}
%   \renewcommand\thefigure{BX}
%\caption[XXXX]{XXXX \cite{Anleitung}}
%\label{Pic:X}}

\subsection{Durchführung}

% describe exp.
XXXX

\subsection{Auswertung}

In diesem Teil wollen wir einfach mit unseren Messwerten und der Formel (\ref{eq:X}) zuerst unsere Werte f\"ur $(s,e)$:
\begin{itemize}
\item $e(80\,\mathrm{mm}):  [35.0+/-0.4242640687119285 23.3+/-0.4242640687119285
 18.9+/-0.4242640687119285 54.39999999999999+/-0.4242640687119285
 44.50000000000001+/-0.4242640687119285]$
\item $s(80\,\mathrm{mm}):  [55.0+/-0.5196152422706632 44.599999999999994+/-0.5196152422706632
 41.2+/-0.5196152422706632 73.2+/-0.5196152422706632
 63.8+/-0.5196152422706632]$
\item $e(80,150\,\mathrm{mm}):  [51.0+/-0.4242640687119285 43.3+/-0.4242640687119285
 40.849999999999994+/-0.4242640687119285 46.6+/-0.4242640687119285
 54.3+/-0.4242640687119285]$
\item $s(80,150\,\mathrm{mm}):  [63.8+/-0.5196152422706632 56.5+/-0.5196152422706632
 54.2+/-0.5196152422706632 59.599999999999994+/-0.5196152422706632
 67.1+/-0.5196152422706632]$
\item $e(80,-200\,\mathrm{mm}):  [32.400000000000006+/-0.4242640687119285
 23.700000000000003+/-0.4242640687119285
 26.799999999999997+/-0.4242640687119285
 43.49999999999999+/-0.4242640687119285 22.5+/-0.4242640687119285]$
\item $s(80,-200\,\mathrm{mm}):  [63.8+/-0.5196152422706632 57.3+/-0.5196152422706632
 67.5+/-0.5196152422706632 73.3+/-0.5196152422706632
 56.3+/-0.5196152422706632]$
\end{itemize}

Wir k\"onnen hier die Rechnungen per Hand mit Gau\ss scher Fehlerfortpflanzung durchf\"uhren. Hierzu m\"ussen wir unsere Gleichung einfach nach jeweils $e$ und $s$ partiell ableiten:

\[
\frac{\partial f}{\partial s}=\frac{s^2+e^2}{4s}
\]
\[
\frac{\partial f}{\partial e}=\frac{-e}{2s}
\]

Dies k\"onnen wir in
\[
\Delta f=\sqrt{\left(\frac{\partial f}{\partial s}\Delta s\right)^2+\left(\frac{\partial f}{\partial e}\Delta e\right)^2}
\]
einsetzen und berechnen. In diesem Fall sind unsere Ergebnissen jedoch mit dem \textit{uncertainties} Paket in Python berechnet worden. \refpy{12} Dieses Paket hat die F\"ahigkeit, Korrelationen zwischen Variablen zu ber\"ucksichtigen \cite{Uncertainties}.

Da uns hier die Mittelwerte interessieren, nutzen wir noch

\mean

f\"ur die Berechnung des Mittelwerts und

\meanstd

f\"ur der Berechnung der Unsicherheit dessen.

Wir erhalten daraus f\"ur die Linse mit $f=80\,$mm $\bar{f}=82\pm1.7\,$mm, f\"ur das System mit $f_1=80\,$mm und $f_2=150\,$mm $\bar{f}=58\pm1.9\,$mm und f\"ur das Linsensystem mit $f_1=80\,$mm und $f_2=200\,$mm $\bar{f}=123\pm1.4\,$mm.

\section{Teil 3}

\subsection{Theorie}

XXXX

\subsection{Aufbau}

% describe set up
% insert pic name, designation, toc caption, caption, label

%\halftime{5}{5}{TEXT}{\fbox{\includegraphics[width=0.5\textwidth]{NAME}}
%   \renewcommand\thefigure{BX}
%\caption[XXXX]{XXXX \cite{Anleitung}}
%\label{Pic:X}}

\subsection{Durchführung}

% describe exp.
XXXX

\subsection{Auswertung}

In diesem Teil wollen wir zuerst mit den Formeln (\ref{eq:beta}), (\ref{eq:g}) und (\ref{eq:b}) $g'$, $b'$, $\beta$ und $\Delta \beta$ bestimmen. Wir erhalten aus unseren Messreihen:

Um dies visuell darzustellen, tragen wir $1+\nicefrac{1}{\beta}$ gegen $g'$ und $1+\beta$ gegen $b'$ dar:

\begin{figure}[h]
\centering
\fbox{\includegraphics[width=0.8\textwidth]{g.png}}
\renewcommand\thefigure{69}
\caption[$1+\nicefrac{1}{\beta}$ gegen $g'$ dargestellt]{$1+\nicefrac{1}{\beta}$ gegen $g'$ dargestellt}
\label{Abb:g}
\end{figure}

\begin{figure}[h]
\centering
\fbox{\includegraphics[width=0.8\textwidth]{b.png}}
\renewcommand\thefigure{420}
\caption[$1+{\beta}$ gegen $b'$ dargestellt]{$1+\beta$ gegen $b'$ dargestellt}
\label{Abb:b}
\end{figure}

Aus der linearen Regression k\"onnen wir $f_1$, $f_2$, $h_1$ und $h_2$ bestimmen. Wir erhalten als Werte:
$f_1 80 200 0.5762491658548258
h_1 80 200 11.03475419102985
f_2 80 200 1.9531933609241
h_2 80 200 11.639603091057374
f_1 200 80 -3.913845161813182
h_1 200 80 11.49900273595246
f_2 200 80 3.2411227934990583
h_2 200 80 11.930724229182056
$

Zur Klarifizierung fertigen wir noch eine (au\ss er der Linsen) ma\ss stabsgetreue Skizze an:\\
\\
\centering{
\begin{tikzpicture}[scale=0.6, inner sep=0]
\def\sx{0.3}

% Optische Achse
\draw[dashed, thick] (-30*\sx,0) -- (40*\sx,0);
% Koordinaten Nullpunkt
\draw (0,-0.2) -- (0,0.2);

% Konkave Linse
\def\xconv{0.5} % x-Koordinate
\draw[thick, fill=yellow!30] 
     ([shift=(190:16cm)]16.5+\xconv,0) arc (190:170:16cm)
     -- ([shift=(10:16cm)] -16.5+\xconv,0)
     arc (10:-10:16cm) 
     -- ([shift=(190:16cm)]16.5+\xconv,0);

% Konvexe Linse
\def\xconx{-1.5} % x-Koordinate
\draw[thick, fill=yellow!30] ([shift=(-15:10cm)]-9+\xconx,0) arc (-15:15:10cm)
 --([shift=(165:10cm)]10+\xconx,0) arc (165:195:10cm)
 --([shift=(-15:10cm)]-9+\xconx,0);

% Gegenstand (Pfeil)
\def\Gx{-24.7*\sx}
\def\Gy{0.7}
\node (G) at (\Gx,\Gy) {};
\draw[->, ultra thick] (\Gx,0) -- (G);%
\node (labG) at (\Gx,\Gy+0.5) {$G$};
% Bild (arrow)
\def\Bx{34.9*\sx}
\def\By{-1.55}
\node (B) at (\Bx,\By) {};
\draw[->, ultra thick] (\Bx,0) -- (B);
\node (labB) at (\Bx,\By-0.5) {$B$};

% Hauptebenen
\def\Hx{5.5*\sx}
\draw (\Hx,-4) -- (\Hx,4); % H1
\node (labH1) at (-\Hx,4.5) {$H_1$};
\draw (-\Hx,-4) -- (-\Hx,4); % H2
\node (labH2) at (\Hx,4.5) {$H_2$};

% Strahlen
% FokusG
\draw[color=blue] (G) -- (-\Hx,\By);
\draw[color=blue] (B) -- (-\Hx,\By);
% FokusB
\draw[color=blue] (G) -- (\Hx,\Gy);
\draw[color=blue] (B) -- (\Hx,\Gy);
%Central
\draw[color=blue] (G) -- (-\Hx,0) -- (\Hx,0)  -- (B);
               
% Label
%Fokus Punkte
\node (F1) at (-18*\sx,-0.5) {$F_1$};
\node (F2) at (14*\sx,0.5) {$F_2$};
% Koordinaten Nullpunkt
\def\cy{-5}
\draw[->] (-30*\sx,\cy) -- (40*\sx,\cy);
\draw (0,-0.2+\cy) -- (0,0.2+\cy);
\draw (10*\sx,-0.2+\cy) -- (10*\sx,0.2+\cy);
\draw (-10*\sx,-0.2+\cy) -- (-10*\sx,0.2+\cy);
\draw (20*\sx,-0.2+\cy) -- (20*\sx,0.2+\cy);
\draw (-20*\sx,-0.2+\cy) -- (-20*\sx,0.2+\cy);
\draw (30*\sx,-0.2+\cy) -- (30*\sx,0.2+\cy);
\draw (-30*\sx,-0.2+\cy) -- (-30*\sx,0.2+\cy);
\node (m30) at (-30*\sx,-1+\cy) {\tiny $-30\,\mathrm{cm}$};
\node (p30) at (30*\sx,-1+\cy) {\tiny $30\,\mathrm{cm}$};
\node (zero) at (0*\sx,-1+\cy) {\tiny $0\,\mathrm{cm}$};

% Gebogener Pfeil
%\draw[->, dashed] (9.5,18) .. controls (4,18) .. (3,11.65);
\end{tikzpicture}}

\section{Teil 4}

\subsection{Theorie}

XXXX

\subsection{Aufbau}

% describe set up
% insert pic name, designation, toc caption, caption, label

%\halftime{5}{5}{TEXT}{\fbox{\includegraphics[width=0.5\textwidth]{NAME}}
%   \renewcommand\thefigure{BX}
%\caption[XXXX]{XXXX \cite{Anleitung}}
%\label{Pic:X}}

\subsection{Durchführung}

% describe exp.
XXXX

\subsection{Auswertung}

XXXX

\section{Diskussion}

XXXX

\pagebreak

\section{Anhang: Tabellen und Diagramme}

\begin{table}[h]
\centering
\caption{XXXX} \vspace{11pt}
$\begin{array}{l}
\textrm{Unsicherheiten:}\\
\textrm{XXXX: } \pm XX \textrm{XX}\\
\end{array}$
\begin{tabular}{ccc}
\toprule
\textrm{XXXX}/\textrm{XX} & \textrm{XXXX}/\textrm{XX} & \textrm{XXXX}/\textrm{XX} \\
\midrule 
2 & 0.26 & 0.23\\
\hline
4 & 0.33 & 0.25\\
\hline 
5 & & 0.3\\
\hline 
6 & 1.25 & 0.83\\
\hline 
8 & 3.9 & 0.83\\ 
\hline
9 & 4.75 & 4.6\\ 
\hline
10 & 4.7 &\\ 
\bottomrule
\end{tabular}
\phantom{$\begin{array}{l}
\textrm{Unsicherheiten:}\\
\textrm{XXXX: } \pm XX \textrm{XX}\\
\end{array}$}
\label{Tab:X}
\end{table}

%\begin{figure}[p]
%\centering
%\fbox{\includegraphics[width=0.8\textwidth]{NAME}}
%\renewcommand\thefigure{BX}
%\caption[XXXX]{XXXX}
%\label{Abb:X}
%\end{figure}

\begin{thebibliography}{9}
\bibitem{Uncertainties}''Correlations between variables are automatically handled, which sets this module apart from many existing error propagation codes.'' - \url{https://pythonhosted.org/uncertainties/}
\bibitem{Anleitung} Physikalisches Institut der Albert-Ludwigs-Universität Freiburg (Hrsg.) (08/2018): Versuchsanleitungen zum Physiklabor für Anfänger*innen, Teil 1, Ferienpraktikum im Sommersemester 2018.
\end{thebibliography}

\end{document}