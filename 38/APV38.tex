\documentclass[11pt,a4paper]{article}

\usepackage[utf8]{inputenc} 
\usepackage[T1]{fontenc} 
\usepackage{lmodern}
\usepackage[margin=2cm]{geometry}
\usepackage[german]{babel}

\setlength{\parindent}{0pt}
\setlength{\parskip}{1ex plus 0.5ex minus 0.5ex}

\usepackage{amsmath} 
\usepackage{graphicx} 
\usepackage{booktabs}
\usepackage[colorlinks]{hyperref}
\usepackage{nicefrac}
\usepackage{gensymb}

\begin{document}

{
\centering 
\large 
Physiklabor für Anf\"anger*innen \\
Ferienpraktikum im Sommersemester 2018 \\[4mm]
\textbf{\LARGE 
Versuch 38: Wärmekapazität
} \\[3mm]
(durchgef\"uhrt am 07.09.2018 bei Daniel Bartle) \\
Andréz Gockel, Patrick M\"unnich\\
\today \\[10mm]
}

\section{Ziel des Versuchs}

Der Versuch ist in zwei Teile geteilt, welche dazu dienen, die Wärmekapazität von Wasser zu bestimmen. Im Teil A bestimmt man die Temperatur Differenz von Wasser während der Umwandlung von mechanischer Energie zu thermischer Energie durch Reibkräfte mit Hilfe des Schürholz-Apparates. Im Teil B verwendet man einen elektrischen Widerstand um elektrische Leistung über ein bestimmten Zeitraum in thermische Energie zu wandeln.

\section{Auswertung und Fehleranalyse}

\subsection{Teil A - Umwandlung von mechanischer Arbeit in Wärme }
$$
\begin{array}{lc}
	\multicolumn{1}{l}{\textrm{Zu Bestimmende Werte}} \\
	\hline
	\textrm{Masse Wasser} & m_W \\
	\textrm{Masse Kalorimeter} & m_{kal} \\
	\textrm{Umdrehung} & n \\
	\textrm{Temperatur} & \Delta T\\
	\textrm{Durchmesser Kalorimeter} & d\\
\end{array}
\qquad
\begin{array}{p{3.5cm}l c}
	\multicolumn{2}{l}{\textrm{Bekannte Werte (Fehler nicht-beitragend)}} \\
	\hline
	\textrm{Spezifische Wärme\-kapazität Kupfer} & c_{Cu} = 0.38\, \nicefrac{\textrm{kJ}}{\textrm{(kgK)}} \\
	\textrm{Wärmekapazität vom Nylonseil} & C_T = 5\, \nicefrac{\textrm{J}}{\textrm{K}} \\
	\textrm{Masse Gewicht} & m = 5\, \textrm{kg} \\
\end{array}
$$



\subsubsection{Aufgabenstellung}

Mit Hilfe von dem Sch\"urholz Apparat ist die W\"armekapazit\"at von Wasser zu bestimmen. Dies wird getan, indem man ein Nylonfaden \"uber ein mit $50$ ml Wasser gef\"ullten Kaloritmeter windet, ein Thermometer dranschraubt und am Ende des Nylonfadens eine $5$ kg Masse dranh\"angt. Das Kalorimeter wird gedreht, sodass die Feder vom Sch\"urholz Apparat entspannt ist, also die Reibkraft $F_R$ die Gewichtskraft $F_G$ ausgleicht. Die Temperatur wird beim Drehen gemessen und notiert.

%Picture from book

\subsubsection{Auswertung}

Die Messung wurde zweimal durchgef\"uhrt. Bei der ersten Durchf\"uhrung wurde die Temperatur\"anderung \"uber 50 Drehungen jeweils im Abstand von 10 Drehungen gemessen. Die zweite Durchf\"uhrung wurde mit 100 Drehungen durchgef\"uhrt und die Temperatur alle 5 Drehungen notiert. Die genauen Messwerte befinden sich im Anhang.

F\"ur die W\"armekapazit\"at gilt:\\

\begin{equation}
C=C_{Kal}+C_T+m_wc_w,\label{eq1}
\end{equation}

was umgestellt werden kann zu:

\begin{equation}
c_w=\frac{C-C_{Kal}-C_T}{m_w}.\label{eq2}
\end{equation}

$C$ wird hier mittels der folgenden Gleichungen bestimmt:

\begin{equation}
C=\frac{Q}{\Delta T}
\end{equation}

\begin{equation}
W_R=mgn\pi d=Q
\end{equation}

Mit unseren Messwerten und dem $uncertainties$ Paket in Python berechnen wir damit:\\
$$\begin{tabular}{|c|c|c|c|}
\hline
\textrm{Messung} & 1 & 2 & \textrm{Mittelwert}\\
\hline
\textrm{W\"armekapazit\"at} $\left[\mathrm{\nicefrac{\mathrm{J}}{K}}\right]$ & $8300\pm1400$ & $3500\pm600$ & $5900\pm800$\\
\hline
\end{tabular}$$

Diese Rechnungen wurden mit dem \textit{uncertainties} Paket in Python durchgef\"uhrt. Diese Rechnungen k\"onnen im Anhang gefunden werden..\\

Der Fehler der Messung mit dem Sch\"urholz Apparat ist aufgrund der gro\ss en Ungenauigkeit der Temperatur und der Anzahl Drehungen sehr gro\ss. Au\ss erdem ist es recht wahrscheinlich, dass $F_{R}$ und $F_{G}$ sich nicht st\"andig ganz ausgleichen, also dadurch auch eine Unsicherheit entsteht. Dies ist einflussreich, da die W\"arme, $Q$, von $F_R$ via $W_R=\int F_Rds$ abh\"angig ist. Da $F_R$ \"uber $F_G$ bestimmt wird und dies bei nicht korrekter Ausgleichung der Beiden nicht akkurat ist, ist also auch $Q$ und dadurch $c_w$ ungenau. Der Literaturwert hierzu ist $4185.5$ $\mathrm{\nicefrac{\mathrm{J}}{K}}$. Der Unterschied ist aufgrund der Messungenauigkeiten und niedrigen Anzahl Messungen sehr gro\ss.


\subsection{Teil B - Umwandlung von elektrischer Arbeit in Wärme}

\subsubsection{Aufgabenstellung}

Zur Bestimmung der W\"armekapazit\"at durch Umwandlung von elektrischer Arbeit in W\"arme nutzt man ein Kalorimeter mit einem Widerstand und Thermometer. Zum Aufw\"armen des Wassers wird der Widerstand an eine Spannungsquelle angeschlo\ss en. Die Temperatur\"anderung wird dann bis zu einem beliebigen Punkt abschnittsweise gemessen. Danach wird gemessen, ab welchem Zeitpunkt die Temperatur wieder abf\"allt.

\subsubsection{Auswertung}

Es wurden zwei Messungen durchgef\"uhrt mit jeweils $116.94\,$g und $113.42\,$g Wasser. Die Wassermenge wurde so gew\"ahlt, damit der Widerstand und das Thermometer das Wasser ber\"uhren. Die Dauern der Messungen waren 38 und 20 Minuten. Diese wurden so gew\"ahlt, dass sie m\"oglichst kurz ausfallen sollten. Temperatur\"anderungen wurden im Abstand von 60 Sekunden gemessen, da diese sonst nicht auff\"allig genug w\"aren, um etwas zu erkennen. Die Messwerte hierzu sind aufgrund ihrer L\"ange im Anhang.



%Extrapolation



%Systematische und statistische Fehler.





%\section{Anhang}


\end{document}