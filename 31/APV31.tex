\documentclass[11pt,a4paper]{article}

\usepackage[utf8]{inputenc} 
\usepackage[T1]{fontenc} 
\usepackage{lmodern}
\usepackage[margin=2cm]{geometry}
\usepackage[german]{babel}
\usepackage{array}
\setlength{\parindent}{0pt}
\setlength{\parskip}{1ex plus 0.5ex minus 0.5ex}
\usepackage{amsmath} 
\usepackage{graphicx} 
\usepackage{booktabs}
\usepackage[colorlinks]{hyperref}
\usepackage{nicefrac}
\usepackage{gensymb}
\usepackage[usenames,dvipsnames,svgnames,table]{xcolor}
\usepackage{tcolorbox}

\hbadness=99999

\newenvironment{supbox}{\begin{tcolorbox}[colback=white,colframe=black,sharp corners,boxrule=.5pt]}{\end{tcolorbox}}
\begin{document}

{
\centering 
\large 
Physiklabor für Anf\"anger*innen \\
Ferienpraktikum im Sommersemester 2018 \\[4mm]
\textbf{\LARGE 
Versuch 31: Mischungsmethode in der Kalorimetrie
} \\[3mm]
(durchgef\"uhrt am 12.09.2018 bei Nico Strauß) \\
Andréz Gockel, Patrick M\"unnich\\
\today \\[10mm]
}

\section{Ziel des Versuchs}

Der Versuch ist in zwei Teile geteilt, welche dazu dienen, mit Hilfe einer geeigneten Wärmeenergiebilanz die Wärmekapazität zu bestimmen. Im Teil A kalibriert man das Messgerät und bestimmt mittels extrapolationsverfahren die Wärmekapazität des Kalorimeters. Für die Messungen wurde ein Temperaturmessfühler, der durch einen DAQ zu einem Komputer verbunden wurde verwendet, dadurch konnten die Messdaten mittels dem Programm LabVIEW gespeichert werden. Im Teil B wurde die Wärmekapazität von zwei Festkörpern bestimmt. 

\section{Auswertung und Fehleranalyse}

\subsection{Teil A - Bestimmung von der Wärmekapazität des Kalorimeters: $C_{kal}$}

 In diesem experiment wurden die Zeiten, Temperaturen und Massen gemessen. Zur kalibrierung haben wir die Temperatur von Eiswasser und kochendem Wasser gemessen.

\begin{figure}[h]
	\centering
	\fbox{\includegraphics[width=0.7\textwidth]{Aufbaoo}}
  	\renewcommand\thefigure{B1}
	\caption{Der Aufbau}
	\label{Bild:1}
\end{figure}




\begin{table}
\centering
$\begin{array}{lr}
	\multicolumn{2}{l}{\textrm{Messwerte zur Bestimmung von }C_{kal}}\\
	\hline
	\textrm{Wasser Masse} & 0\,\textrm{g} \\
	\textrm{Temperatur Wasser 1} & 0\,\celsius \\
	\textrm{Temperatur Wasser 2} & 100\,\celsius \\
	\textrm{Temperatur nach Mischen} & 50\,\celsius\\
	\hline
\end{array}$
\renewcommand\thetable{T1}
\caption{Messwerte für Teil A}
\label{tab:1}
\end{table}


\subsubsection{Aufgabenstellung}


\subsubsection{Auswertung}

Die Messung wurde zweimal durchgef\"uhrt. Bei der ersten Durchf\"uhrung wurde die Temperatur\"anderung \"uber 50 Drehungen jeweils im Abstand von 10 Drehungen gemessen. Die zweite Durchf\"uhrung wurde mit 100 Drehungen durchgef\"uhrt und die Temperatur alle 5 Drehungen notiert. Die genauen Messwerte befinden sich im Anhang. \ref{table:m1}, \ref{table:m2}.

Die restlichen Messungen ergaben: $$m_W = 79.18(3)\textrm{g}, \quad m_{kal} = 98.05(3)\textrm{g}, \quad d = 4.765(3)\textrm{cm}$$

F\"ur die W\"armekapazit\"at gilt:\\

\begin{equation}
C=C_{Kal}+C_T+m_wc_w,\label{eq1}
\end{equation}

was umgestellt werden kann zu:

\begin{equation}
c_w=\frac{C-C_{Kal}-C_T}{m_w}.\label{eq2}
\end{equation}

$C$ wird hier mittels der folgenden Gleichungen bestimmt:

\begin{equation}
C=\frac{Q}{\Delta T}
\end{equation}

\begin{equation}
W_R=mgn\pi d=Q
\end{equation}

Mit unseren Messwerten und dem $uncertainties$ Paket in Python berechnen wir damit:\\
$$\begin{tabular}{|c|c|c|c|c|}
\hline
\textrm{Messung} & 1 & 2 & \textrm{Mittelwert} & \textrm{Gewichtet} \\ 
\hline
\textrm{W\"armekapazit\"at} $\left[\mathrm{\nicefrac{\mathrm{J}}{kg K}}\right]$ & $8300\pm1400$ & $3500\pm600$ & $5900\pm800$ & $4187\pm 554$ \\
\hline
\end{tabular}$$

Diese Rechnungen wurden mit dem \textit{uncertainties} Paket in Python durchgef\"uhrt. Diese Rechnungen k\"onnen im Anhang gefunden werden..\\

Der Fehler der Messung mit dem Sch\"urholz Apparat ist aufgrund der gro\ss en Ungenauigkeit der Temperatur und der Anzahl Drehungen sehr gro\ss. Au\ss erdem ist es recht wahrscheinlich, dass $F_{R}$ und $F_{G}$ sich nicht st\"andig ganz ausgleichen, also dadurch auch eine Unsicherheit entsteht. Dies ist einflussreich, da die W\"arme, $Q$, von $F_R$ via $W_R=\int F_Rds$ abh\"angig ist. Da $F_R$ \"uber $F_G$ bestimmt wird und dies bei nicht korrekter Ausgleichung der Beiden nicht akkurat ist, ist also auch $Q$ und dadurch $c_w$ ungenau. Der Literaturwert hierzu ist $4182$ $\mathrm{\nicefrac{\mathrm{J}}{kg K}}$. Der Unterschied ist aufgrund der Messungenauigkeiten und niedrigen Anzahl Messungen sehr gro\ss.

\pagebreak

\subsection{Teil B - Umwandlung von elektrischer Arbeit in Wärme}

\subsubsection{Aufgabenstellung}

Zur Bestimmung der W\"armekapazit\"at durch Umwandlung von elektrischer Arbeit in W\"arme nutzt man ein Kalorimeter mit einem Widerstand und Thermometer. Zum Aufw\"armen des Wassers wird der Widerstand an eine Spannungsquelle angeschlo\ss en. Die Temperatur\"anderung wird dann bis zu einem beliebigen Punkt abschnittsweise gemessen. Danach wird gemessen, ab welchem Zeitpunkt die Temperatur wieder abf\"allt.

\subsubsection{Auswertung}
\begin{supbox}
Es wurden zwei Messungen durchgef\"uhrt mit jeweils $116.94\,$g und $113.42\,$g Wasser. Die Wassermenge wurde so gew\"ahlt, damit der Widerstand und das Thermometer in dem Wasser eingetaucht sind. Die Dauern der Messungen waren 38 und 20 Minuten. Diese wurden so gew\"ahlt, dass sie m\"oglichst kurz ausfallen sollten. Temperatur\"anderungen wurden im Abstand von 60 Sekunden gemessen, da diese sonst nicht auff\"allig genug w\"aren, um etwas zu erkennen. Die Messwerte hierzu sind aufgrund ihrer L\"ange im Anhang.

Das extrapolationsverfahren der 1. Messreihe ergibt $T_{max} = 26.5\celsius$ da der Temperaturabfall erst nach 30 begann. Für die 2. Messreihe ergibt das extrapolationsverfahren $T_{max} = 41.45\celsius$ \ref{Dia:1}, \ref{Ext}
\end{supbox}
\begin{table}[h!]
	\centering
	\rowcolors{2}{gray!10}{white}
	\begin{tabular}{|c|cccc|}
		\multicolumn{5}{c}{\textrm{Extrapolationverfahren}} \\
		\noalign{\global\arrayrulewidth=0.4mm}
		\hline
		\noalign{\global\arrayrulewidth=0.2mm}
		\textrm{Messreihe} & $a$ in \celsius & $u_a$ in \celsius & b in $\nicefrac{\celsius}{\textrm{s}}$ & $u_b$ in $\nicefrac{\celsius}{\textrm{s}}$ \\
		\hline
	1 & 26.5 & 0.037 &  -33.579 & 2.105 $\times 10^{-5}$ \\
	2 & 42.65 & 1.605 & -0.0025 & 0.001443 \\
		\hline
	\end{tabular}
	\renewcommand\thetable{T3}
	\caption{Wertetablle für die extrapolation}
	\label{Ext}
\end{table}




%Systematische und statistische Fehler.




\section{Anhang}

\begin{figure}[p]
\centering
\fbox{\includegraphics[width=0.7\textwidth]{extrapDia}}
   \renewcommand\thefigure{A1}
\caption{Extrapolation 2. Messreihe}
\label{Dia:1}
\end{figure}


\begin{table}[p]
	\centering
	\rowcolors{2}{gray!10}{white}
	\begin{tabular}{|r|l|}
		\multicolumn{2}{c}{\textrm{Messreihe 1}} \\
		\noalign{\global\arrayrulewidth=0.4mm}
		\hline
		\noalign{\global\arrayrulewidth=0.2mm}
		\textrm{Rotationen }$n \pm 0.3$ & \textrm{Temperatur }$T \pm 0.05\celsius$\\
		\hline
		0 & 24 \\
		10 & 24.1 \\
		20 & 24.3 \\
		30 & 24.5 \\
		40 & 24.6 \\
		50 & 24.8 \\
		\hline
	\end{tabular}
	\renewcommand\thetable{T1}
	\caption{Messreihe 1 für den ersten Versuchsteil}
	\label{table:m1}
\end{table}

\begin{table}[p]
	\centering
	\rowcolors{2}{gray!10}{white}
	\begin{tabular}{|r|l|}
		\multicolumn{2}{c}{\textrm{Messreihe 2}} \\
		\noalign{\global\arrayrulewidth=0.4mm}
		\hline
		\noalign{\global\arrayrulewidth=0.2mm}
		\textrm{Rotationen }$n \pm 0.3$ & \textrm{Temperatur }$T \pm 0.05\celsius$\\
		\hline
		0 & 24.3 \\
		5 & 24.3 \\
		10 & 24.4 \\
		15 & 24.5 \\
		20 & 24.6 \\
		25 & 24.7 \\
		30 & 24.8 \\
		35 & 24.9 \\
		40 & 25 \\
		45 & 25.1 \\
		50 & 25.2 \\
		55 & 25.2 \\
		60 & 25.4 \\
		65 & 25.5 \\
		70 & 25.5 \\
		75 & 25.6 \\
		80 & 25.6 \\
		85 & 25.7 \\
		90 & 25.8 \\
		95 & 26 \\
		100 & 26 \\
		\hline
	\end{tabular}
	\renewcommand\thetable{T2}
	\caption{Messreihe 2 für den ersten Versuchsteil}
	\label{table:m2}
\end{table}

\begin{table}[p]
\centering
$\begin{array}{rl}
\multicolumn{2}{c}{\textrm{\underline{Unsicherheiten:}}}\\
\textrm{Zeit: } & \pm 0.03 \textrm{s}\\
\textrm{Temperatur: } & \pm 0.02 \textrm{\celsius}\\
\textrm{Strom: } & \pm 0.03 \textrm{A}\\
\textrm{Spannung: } & \pm 0.02 \textrm{V}
\end{array}$
\rowcolors{2}{gray!10}{white}
\begin{tabular}{|c|c|c|c|}
\multicolumn{4}{l}{Wasser 116.94(3)\,g}\\
\hline
$t$ in s & $T$ in $^\circ\textrm{C}$ & $I$ in A & $U$ in V \\
\hline 
0   & 22   & 1.5 & 14.9 \\
60  & 22   & 1.5 & 14.9 \\
120 & 23   & 1.5 & 14.9 \\
180 & 24.5 & 0   & 0    \\
240 & 26.3 & 0   & 0    \\ 
300 & 26.5 & 0   & 0    \\ 
360 & 26.5 & 0   & 0	\\ 
$\vdots$ & $\vdots$ & $\vdots$ & $\vdots$ \\
2280 & 26.4 & 0 & 0 \\
\hline
\end{tabular}
\phantom{$\begin{array}{rl}
\multicolumn{2}{l}{\textrm{\underline{Unsicherheiten:}}}\\
\textrm{Zeit: } & \pm 0.03 \textrm{s}\\
\textrm{Temperatur: } & \pm 0.02 \textrm{\celsius}\\
\textrm{Strom: } & \pm 0.03 \textrm{A}\\
\textrm{Spannung: } & \pm 0.02 \textrm{V}
\end{array}$
}
\renewcommand\thetable{T4}
\caption{1. Messwerte für Teil B}
\label{tab:B1}
\end{table}

\begin{table}[p]
\centering
$\begin{array}{rl}
\multicolumn{2}{l}{\textrm{\underline{Unsicherheiten:}}}\\
\textrm{Zeit: } & \pm 0.03 \textrm{s}\\
\textrm{Temperatur: } & \pm 0.02 \textrm{\celsius}\\
\textrm{Strom: } & \pm 0.03 \textrm{A}\\
\textrm{Spannung: } & \pm 0.02 \textrm{V}
\end{array}$
\rowcolors{2}{gray!10}{white}
\begin{tabular}{|c|c|c|c|}
\multicolumn{4}{l}{Wasser 113.42(3)\,g}\\
\hline
$t$ in s & $T$ in $^\circ\textrm{C}$ & $I$ in A & $U$ in V \\
\hline 
0   & 22 & 1.5 & 14.9\\
60  & 22 & 1.5 & 14.9\\
120 & 23 & 1.5 & 14.9\\
180 & 24 & 1.5 & 14.9\\
240 & 26 & 1.5 & 14.9\\ 
300 & 27 & 1.5 & 14.9\\ 
360 & 28 & 1.5 & 14.9\\ 
420 & 29.2 & 1.5 & 14.9\\ 
480 & 30.3 & 1.5 & 14.9\\ 
540 & 31.9 & 1.5 & 14.9\\ 
600 & 33 & 1.5 & 14.9\\ 
660 & 33.7 & 1.5 & 14.9\\ 
720 & 35 & 1.5 & 14.9\\ 
780 & 35 & 1.5 & 14.9\\ 
840 & 36 & 1.5 &14.9\\
900 & 37 & 1.5 & 14.9\\
960 & 38.2 & 0 & 0\\
1020 & 39.5 & 0 & 0\\
1080 & 40 & 0 & 0\\
1140 & 40 & 0 & 0\\
1200 & 39.5 & 0 & 0\\
\hline
\end{tabular}
\phantom{$\begin{array}{rl}
\multicolumn{2}{l}{\textrm{\underline{Unsicherheiten:}}}\\
\textrm{Zeit: } & \pm 0.03 \textrm{s}\\
\textrm{Temperatur: } & \pm 0.02 \textrm{\celsius}\\
\textrm{Strom: } & \pm 0.03 \textrm{A}\\
\textrm{Spannung: } & \pm 0.02 \textrm{V}
\end{array}$
}
\renewcommand\thetable{T5}
\caption{2. Messwerte für Teil B}
\label{tab:B2}
\end{table}


\end{document}